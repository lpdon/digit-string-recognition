\section{Connectionist Temporal Classification}\label{sec:ctc}

The loss is calculated using PyTorch's CTC loss function \cite{pytorch_ctc},
which was first introduced by Graves et al.\cite{CTC}. It can be considered as
an output layer to solve the following main problems:
\begin{itemize}
%\itemsep -1.3mm
\item The consumed time during the annotation process of the whole data set on
digit level, which would be one to one correspondence between outputs and
labels.
\item The limitation on the available information about the digits and its
constraints which could be affected during the processing through many factors
like noise or image resolution.
\end{itemize}
These issues can be avoided by using a function which is provided with the
output matrix of the neural network and the corresponding ground-truth (GT) text
by trying all possible digit alignments between inputs and labels in the image.
It outputs at the end a classification at each input value by taking the highest
probability. CTC uses the network to label the entire input sequence at once.
This means the network can be trained with an unsegmented dataset and the final
label sequence can be seen directly from the network output. 

The input of CTC is
a sequence $y=y_{1},....y_{T}$ from a recurrent neural network with $m$ inputs
and $n$ outputs, where $T$ is the sequence length. For instance for an input
sequence $y$ of length $T$, defined a RNN with $m$ inputs, $n$ outputs an
weight vector $w$ as a continuous map $N_w$: ($\mathbb{R}^m$) $\rightarrow$
($\mathbb{R}^n$). The first step called encoding can be initiated through
inserting randomly a pseudo character called 'blank' denoted with {-} which must
be inserted both in case of digits duplication and areas without digits. For a
given $y_{T}$ it gives an output distribution over all possibilities. A
conditional probability is defined as the sum of probabilities of all $\pi$
which are mapped by $F$ onto $I$:
\begin{equation}
p( I | y)=\sum_{\pi \in F(I)} p( \pi | y )
\label{eq1}
\end{equation}
where the conditional probability of $\pi$ is defined as: 
\begin{equation}
p( \pi | y ) =\prod_t^T y_{\pi_{t}}
\label{eq2}
\end{equation}
The probability $p( \pi | y )$ of a particular path or respectively label
observation is the product of all the softmax 'digit scores' outputs $y_{\pi_{t}}$
over time $T$ for one alignment. The function takes the negative log probability
of ground truth all the training examples in training set $D$:
\begin{equation}
\sum_{(I,y) \in D} - \log  p( I | y )
\label{eq3}
\end{equation}
where $y$ is the sequence produced by the recurrent layers from $x$. By
decoding, the function calculates the best path by taking the most likely
character per time-step and removes the changes introduced by encoding by first
removing duplicate characters and then removing all blanks from the path. As
a result remains the recognized text.

\section{Recurrent Neural Network}\label{sec:rnn}
Based on the human fact that we do not need to rethink from scratch rather
building up on the available results, Recurrent Neural Network has been
developed to solve this issue with the normal neural network using connected
chunks of neural network. The information can be passed over the chunks in a
loop intimately, but It shouldn’t go so far back to avoid gradient vanish
problem during the parameter optimization especially by a small number of
Iterations. So LSTM should be used instead of the RNN.The LSTM short of learning
long-term dependencies does have the ability to remove or add information to the
cell state, carefully regulated by 4 specified mechanism called gates:
\begin{itemize}
%\itemsep -1.3mm
\item Forgetting mechanism.
\item Saving (remembering) mechanism.
\item Learning mechanism.
\item Focusing long-term memory into working memory which is the next short memory.
\end{itemize}
LSTM transforms its memory in a very precise way using these mechanisms for
which pieces of information to remember, which to update, and which to learn.
This helps it keep track of information over longer periods of time. At the same
time, the future events should be also considered, therefore the network will be
handle that trainings in the opposite direction, i.e each training sequence
forwards and backwards to two separate LSTM layer, und that has been called
bidirectional LSTM. 